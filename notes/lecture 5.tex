\documentclass{article}
\usepackage{amsmath}
\usepackage{algorithm}
\usepackage{algorithmic}
\usepackage{graphicx}
\usepackage{program}
\usepackage{mathtools}
\usepackage{float}
\usepackage{fancyvrb}
\usepackage{amsfonts}
\newcommand{\overbar}[1]{\mkern 1.5mu\overline{\mkern-1.5mu#1\mkern-1.5mu}\mkern 1.5mu}

\DeclarePairedDelimiter\floor{\lfloor}{\rfloor}
\begin{document}
		\begin{equation}
		F_{8} = (F_2 \otimes I_4)T^{8}_4(I_4 \otimes F_2) L^{8}_4
	\end{equation}
	\begin{equation}
		F_8=(F_2 \otimes I_4)T^8_4(I_2 \otimes F_2 \otimes I_2)(I_2 \otimes T^4_2)(I_4 \otimes F_2)
	\end{equation}
	\begin{equation}
		A \otimes B = L(B\otimes A)L'
	\end{equation}
	So we can flip tensor products by some permutation matrix
	
	$L^4_2$ is loaded stride inverse of storing the stride. Then, we can write everything using $I_4 \otimes F_2$
	
	Now, how do we find these permutaions? one to one onto mapping on a finite set.
	
	They are one to one permutations so there are $n!$ permutations. We can combine permutation operations by multiplying them. Not communicative
	
	Given we have matrices
	
	\begin{equation}
		\sigma = 
		\begin{pmatrix}
			0 &  1 & 2 & 3 \\
			1 & 2 & 3 & 0
		\end{pmatrix}
	\end{equation}
	\begin{equation}
		t = 
		\begin{pmatrix}
			0 &  1 & 2 & 3 \\
			1 & 0 & 3 & 2
		\end{pmatrix}
	\end{equation}
	$\sigma$ is the same as
	\begin{equation}
		P=\begin{pmatrix}
			0 & 1 & 0 & 0 \\
			1 & 0 & 0 & 0 \\
			0 & 0 & 0 & 1 \\
			0 & 0 & 1 & 0
		\end{pmatrix}
	\end{equation}
	
	Same multiplications can be done. If i goes to j, the matrix [i][j] is 1. For the inverse, if i goes to j, j must go to i. So given the permutation is just the transpose.
	
	We showed one to one isomorphic mapping from the idea of permutations of mapping indices to matrices which preserves all the properties.
	
	\begin{equation}
		D=\begin{pmatrix}
			d_0 & 0 & 0 & 0 \\
			0 & d_1 & 0 & 0 \\
			0 & 0 & d_2 & 0 \\
			0 & 0 & 0 & d_3
		\end{pmatrix}
	\end{equation}
	What is $PDP^T$
	\begin{align}
		\begin{pmatrix}
			0 & 1 & 0 & 0 \\
			1 & 0 & 0 & 0 \\
			0 & 0 & 0 & 1 \\
			0 & 0 & 1 & 0
		\end{pmatrix}
		\begin{pmatrix}
			d_0 & 0 & 0 & 0 \\
			0 & d_1 & 0 & 0 \\
			0 & 0 & d_2 & 0 \\
			0 & 0 & 0 & d_3
		\end{pmatrix}
		\begin{pmatrix}
			0 & 1 & 0 & 0 \\
			1 & 0 & 0 & 0 \\
			0 & 0 & 0 & 1 \\
			0 & 0 & 1 & 0
		\end{pmatrix}
		&=
		\begin{pmatrix}
			d_1 & 0 & 0 & 0 \\
			0 & d_0 & 0 & 0 \\
			0 & 0 & d_3 & 0 \\
			0 & 0 & 0 & d_2
		\end{pmatrix}
	\end{align}
	conjugation of a diagonal matrix. As 0 goes to 1, 1 goes to 0. This is changing basis to the permutation matrix
	
	\begin{equation}
		A_m \otimes B_n = L^{mn}_m (B_n \otimes A_m) L^{mn}_n
	\end{equation}
	As
	$L^{mn}_n$ permutes the input at stride n so
	\begin{equation}
		(L^{mn}_n)^{-1} = L^{mn}_m
	\end{equation}
	Then we can do
	\begin{equation}
		(A_m \otimes B_n) L^{mn}_m = L^{mn}_m (B_n \otimes A_m) 
	\end{equation}

	L is the opposite of the tensor.
	
	\begin{equation}
		(I_2 \otimes F_2) \otimes I_2 = L^8_4(I_4 \otimes F_2)L^8_2
	\end{equation}
	
	\begin{equation}
		L^8_4 L^8_4 = L^8_2 L^8_2 L^8_4 = L^8_2
	\end{equation}
	
	\begin{equation}
		F_8=L^8_2 (I_4 \otimes F_2)L^8_4 T^8_4 L^8_4(I_4 \otimes F_2)L^8_2 (I_2 \otimes T^4_2)(I_4 \otimes F_2)R_8
	\end{equation}
	\begin{equation}
		F_8=L^8_2 (I_4 \otimes F_2)\overbar{T^{8}_4} L^8_2(I_4 \otimes F_2) \overbar{(I_2 \otimes T^4_2)}L^8_2(I_4 \otimes F_2)R_8
	\end{equation}

	Now what is $\overbar{T^{8}_4}$ and $\overbar{(I_2 \otimes T^4_2)}$?
	
	Hint, the $e$ stuff
	
	Prove 
	
	\begin{equation}
		A_m \otimes B_n = L^{mn}_m (B_n \otimes A_m) L^{mn}_n
	\end{equation}
	
	\begin{equation}
		(A \otimes B)(x \otimes y)
	\end{equation}
	For $A_m \otimes B_n$, is a mn size matrix. At index i, j, the value is $A_m[i//n+i \mod m]$
\end{document}
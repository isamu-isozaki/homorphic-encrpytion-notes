\documentclass{article}
\usepackage{amsmath}
\usepackage{algorithm}
\usepackage{algorithmic}
\usepackage{graphicx}
\usepackage{program}
\usepackage{mathtools}
\usepackage{float}
\usepackage{fancyvrb}
\usepackage{amsfonts}
\newcommand{\overbar}[1]{\mkern 1.5mu\overline{\mkern-1.5mu#1\mkern-1.5mu}\mkern 1.5mu}

\DeclarePairedDelimiter\floor{\lfloor}{\rfloor}
\begin{document}
	\begin{equation}
		(I_4 \otimes F_2)R_8=R_8(F_2 \otimes I_4)
	\end{equation}
	\begin{align}
		(I_4 \otimes F_2)R_8(e^2_i \otimes e^2_j \otimes e^2_k)&=(I_4 \otimes F_2)(e^4_{2k+j} \otimes e^2_i) \\
		&=(e^4_{2k+j} \otimes F_2e^2_i)
	\end{align}
	\begin{align}
		R_8(F_2 \otimes I_4)(e^2_i \otimes e^2_j \otimes e^2_k)
	\end{align}
	I see
	\begin{equation}
		R_8 =(R_4 \otimes I_2)L^8_4=(L^4_2 \otimes I_2)L^8_4
	\end{equation}
	
	\begin{equation}
		F_{rs} = (F_r \otimes I_s)T^{rs}_s(I_r \otimes F_s) L^{rs}_r
	\end{equation}
	so
	\begin{equation}
		F_{16}=(F_2 \otimes I_8)T^{16}_8(I_2 \otimes F_8)R_{16}
	\end{equation}
	Now as
	\begin{equation}
		F_8=(F_2 \otimes I_4) T^8_4(I_2 \otimes F_2 \otimes I_2)(I_2 \otimes T^4_2)(I_4 \otimes F_2)R_8
	\end{equation}
	We can do
	\begin{align}
		F_8&=(F_2 \otimes I_4) T^8_4(I_2 \otimes F_2 \otimes I_2)(I_2 \otimes T^4_2)R_8(F_2 \otimes I_4)\\
		&=(F_2 \otimes I_4) T^8_4R_8(F_2 \otimes I_4)(T^4_2 \otimes I_2)(F_2 \otimes I_4)
	\end{align}
\begin{equation}
	(R_n \otimes I_2)L^{2n}_n = R_{2n}
\end{equation}
\end{document}
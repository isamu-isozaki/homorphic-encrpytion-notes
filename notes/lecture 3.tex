\documentclass{article}
\usepackage{amsmath}
\usepackage{algorithm}
\usepackage{algorithmic}
\usepackage{graphicx}
\usepackage{program}
\usepackage{mathtools}
\usepackage{float}
\usepackage{fancyvrb}
\usepackage{amsfonts}
\newcommand{\overbar}[1]{\mkern 1.5mu\overline{\mkern-1.5mu#1\mkern-1.5mu}\mkern 1.5mu}

\DeclarePairedDelimiter\floor{\lfloor}{\rfloor}
\begin{document}
	\begin{equation}
		\begin{pmatrix}
			F_2 & W_2F_2 \\
			F_2 & -W_2F_2
		\end{pmatrix}
	\end{equation}
	\begin{equation}
		=\begin{pmatrix}
			1 & 1 & 1 & 1 \\
			1 & \omega & \omega^2 & \omega^3 \\
			1 & \omega^2 & 1 & \omega^2 \\
			1 & \omega^3 & \omega^2 & \omega
		\end{pmatrix}
		\begin{pmatrix}
			1 & 0 & 0 & 0 \\
			0 & 0 & 1 & 0 \\
			0 & 1 & 0 & 0 \\
			0 & 0 & 0 & 1
		\end{pmatrix}
	\end{equation}
	\begin{equation}
		F_n=(F_2 \otimes I_m)T_m^{2m}(I_2 \otimes F_m)L_m^{2m}
	\end{equation}
	As $(L_m^{2m})^{-1}$ is the inverse
	\begin{equation}
		F_nL_m^{2m}=(F_2 \otimes I_m)T_m^{2m}(I_2 \otimes F_m)
	\end{equation}
	$L_m^{2m}$ takes stride 2(just the columns). What is this?
	\begin{equation}
		W_m = \begin{pmatrix}
			1 & .... \\
			0 & \omega \\
		\end{pmatrix}
	\end{equation}
etc.
	\begin{equation}
		\begin{pmatrix}
			\omega^{i2j} & \omega^{i(2j+1)} \\
			\omega^{(i+m)2j} & \omega^{(i+m)(2j+1)}
		\end{pmatrix}
		=
		\begin{pmatrix}
			\omega^{i2j} & \omega^{2ij}\omega^i \\
			\omega^{2ij}\omega^{2mj} & \omega^{2ij}\omega^{2mj}\omega^{m}\omega^i
		\end{pmatrix}
	\end{equation}

	This is
	\begin{equation}
		\begin{pmatrix}
			F_m & W_mF_m \\
			F_m & -W_mF_m
		\end{pmatrix}
		=
		\begin{pmatrix}
			I_m & I_m \\
			I_m & -I_m
		\end{pmatrix}
		\begin{pmatrix}
			I_m & 0 \\
			0 & W_m
		\end{pmatrix}
			\begin{pmatrix}
				F_m & 0 \\
				0 & F_m
			\end{pmatrix}
	\end{equation}
	Thus proven
	
	
	Notes
	
\end{document}